\PassOptionsToPackage{unicode=true}{hyperref} % options for packages loaded elsewhere
\PassOptionsToPackage{hyphens}{url}
%
\documentclass[
]{article}
\usepackage{lmodern}
\usepackage{amssymb,amsmath}
\usepackage{ifxetex,ifluatex}
\ifnum 0\ifxetex 1\fi\ifluatex 1\fi=0 % if pdftex
  \usepackage[T1]{fontenc}
  \usepackage[utf8]{inputenc}
  \usepackage{textcomp} % provides euro and other symbols
\else % if luatex or xelatex
  \usepackage{unicode-math}
  \defaultfontfeatures{Scale=MatchLowercase}
  \defaultfontfeatures[\rmfamily]{Ligatures=TeX,Scale=1}
\fi
% use upquote if available, for straight quotes in verbatim environments
\IfFileExists{upquote.sty}{\usepackage{upquote}}{}
\IfFileExists{microtype.sty}{% use microtype if available
  \usepackage[]{microtype}
  \UseMicrotypeSet[protrusion]{basicmath} % disable protrusion for tt fonts
}{}
\makeatletter
\@ifundefined{KOMAClassName}{% if non-KOMA class
  \IfFileExists{parskip.sty}{%
    \usepackage{parskip}
  }{% else
    \setlength{\parindent}{0pt}
    \setlength{\parskip}{6pt plus 2pt minus 1pt}}
}{% if KOMA class
  \KOMAoptions{parskip=half}}
\makeatother
\usepackage{xcolor}
\IfFileExists{xurl.sty}{\usepackage{xurl}}{} % add URL line breaks if available
\IfFileExists{bookmark.sty}{\usepackage{bookmark}}{\usepackage{hyperref}}
\hypersetup{
  pdftitle={hw3-p2},
  pdfauthor={Mark Pearl},
  pdfborder={0 0 0},
  breaklinks=true}
\urlstyle{same}  % don't use monospace font for urls
\usepackage[margin=1in]{geometry}
\usepackage{color}
\usepackage{fancyvrb}
\newcommand{\VerbBar}{|}
\newcommand{\VERB}{\Verb[commandchars=\\\{\}]}
\DefineVerbatimEnvironment{Highlighting}{Verbatim}{commandchars=\\\{\}}
% Add ',fontsize=\small' for more characters per line
\usepackage{framed}
\definecolor{shadecolor}{RGB}{248,248,248}
\newenvironment{Shaded}{\begin{snugshade}}{\end{snugshade}}
\newcommand{\AlertTok}[1]{\textcolor[rgb]{0.94,0.16,0.16}{#1}}
\newcommand{\AnnotationTok}[1]{\textcolor[rgb]{0.56,0.35,0.01}{\textbf{\textit{#1}}}}
\newcommand{\AttributeTok}[1]{\textcolor[rgb]{0.77,0.63,0.00}{#1}}
\newcommand{\BaseNTok}[1]{\textcolor[rgb]{0.00,0.00,0.81}{#1}}
\newcommand{\BuiltInTok}[1]{#1}
\newcommand{\CharTok}[1]{\textcolor[rgb]{0.31,0.60,0.02}{#1}}
\newcommand{\CommentTok}[1]{\textcolor[rgb]{0.56,0.35,0.01}{\textit{#1}}}
\newcommand{\CommentVarTok}[1]{\textcolor[rgb]{0.56,0.35,0.01}{\textbf{\textit{#1}}}}
\newcommand{\ConstantTok}[1]{\textcolor[rgb]{0.00,0.00,0.00}{#1}}
\newcommand{\ControlFlowTok}[1]{\textcolor[rgb]{0.13,0.29,0.53}{\textbf{#1}}}
\newcommand{\DataTypeTok}[1]{\textcolor[rgb]{0.13,0.29,0.53}{#1}}
\newcommand{\DecValTok}[1]{\textcolor[rgb]{0.00,0.00,0.81}{#1}}
\newcommand{\DocumentationTok}[1]{\textcolor[rgb]{0.56,0.35,0.01}{\textbf{\textit{#1}}}}
\newcommand{\ErrorTok}[1]{\textcolor[rgb]{0.64,0.00,0.00}{\textbf{#1}}}
\newcommand{\ExtensionTok}[1]{#1}
\newcommand{\FloatTok}[1]{\textcolor[rgb]{0.00,0.00,0.81}{#1}}
\newcommand{\FunctionTok}[1]{\textcolor[rgb]{0.00,0.00,0.00}{#1}}
\newcommand{\ImportTok}[1]{#1}
\newcommand{\InformationTok}[1]{\textcolor[rgb]{0.56,0.35,0.01}{\textbf{\textit{#1}}}}
\newcommand{\KeywordTok}[1]{\textcolor[rgb]{0.13,0.29,0.53}{\textbf{#1}}}
\newcommand{\NormalTok}[1]{#1}
\newcommand{\OperatorTok}[1]{\textcolor[rgb]{0.81,0.36,0.00}{\textbf{#1}}}
\newcommand{\OtherTok}[1]{\textcolor[rgb]{0.56,0.35,0.01}{#1}}
\newcommand{\PreprocessorTok}[1]{\textcolor[rgb]{0.56,0.35,0.01}{\textit{#1}}}
\newcommand{\RegionMarkerTok}[1]{#1}
\newcommand{\SpecialCharTok}[1]{\textcolor[rgb]{0.00,0.00,0.00}{#1}}
\newcommand{\SpecialStringTok}[1]{\textcolor[rgb]{0.31,0.60,0.02}{#1}}
\newcommand{\StringTok}[1]{\textcolor[rgb]{0.31,0.60,0.02}{#1}}
\newcommand{\VariableTok}[1]{\textcolor[rgb]{0.00,0.00,0.00}{#1}}
\newcommand{\VerbatimStringTok}[1]{\textcolor[rgb]{0.31,0.60,0.02}{#1}}
\newcommand{\WarningTok}[1]{\textcolor[rgb]{0.56,0.35,0.01}{\textbf{\textit{#1}}}}
\usepackage{graphicx,grffile}
\makeatletter
\def\maxwidth{\ifdim\Gin@nat@width>\linewidth\linewidth\else\Gin@nat@width\fi}
\def\maxheight{\ifdim\Gin@nat@height>\textheight\textheight\else\Gin@nat@height\fi}
\makeatother
% Scale images if necessary, so that they will not overflow the page
% margins by default, and it is still possible to overwrite the defaults
% using explicit options in \includegraphics[width, height, ...]{}
\setkeys{Gin}{width=\maxwidth,height=\maxheight,keepaspectratio}
\setlength{\emergencystretch}{3em}  % prevent overfull lines
\providecommand{\tightlist}{%
  \setlength{\itemsep}{0pt}\setlength{\parskip}{0pt}}
\setcounter{secnumdepth}{-2}
% Redefines (sub)paragraphs to behave more like sections
\ifx\paragraph\undefined\else
  \let\oldparagraph\paragraph
  \renewcommand{\paragraph}[1]{\oldparagraph{#1}\mbox{}}
\fi
\ifx\subparagraph\undefined\else
  \let\oldsubparagraph\subparagraph
  \renewcommand{\subparagraph}[1]{\oldsubparagraph{#1}\mbox{}}
\fi

% set default figure placement to htbp
\makeatletter
\def\fps@figure{htbp}
\makeatother


\title{hw3-p2}
\author{Mark Pearl}
\date{4/15/2020}

\begin{document}
\maketitle

\hypertarget{graded-homework-3-part-2}{%
\subsection{Graded Homework \#3: Part
2}\label{graded-homework-3-part-2}}

Homework 3 - Part 2 is from Week 11 and 12 and weighs 4\% of your grade.
It will require you to submit one file (HTML or PDF) which will be peer
corrected by three of your peers. The TAs will also go through your
submission and eventually assign the final marks.

Submit ONE HTML or PDF file with your code and answers to these 8
questions neatly labeled, clear and concise. Use RMarkdown to knit the
file. You may work on the homework for as long as you like within the
given window. As long as you do not click submit, you can enter and exit
the assignment as many times as necessary during the time period that it
is available. Again, please note, you should only click ``submit'' when
you are completely finished with the assignment and ready to submit it
for grading.

Also, please remember that you are to complete this assignment on your
own. Any help given or received constitutes cheating. If you have any
general questions about the assignment, please post it to the Piazza
board. If your question involves specific references to the answer to a
question or questions, please be sure to mark your post as private.

Instructions for Q.1 to 4

Please use the Facebook Ad dataset KAG.csv (Links to an external site.)
for the next set of questions. We advise solving these questions using R
(preferably using dplyr library wherever applicable) after reviewing the
code provided for Week 11 and other resources provided for learning
dplyr in R Learning Guide.

\begin{Shaded}
\begin{Highlighting}[]
\NormalTok{kag_csv <-}\StringTok{ }\KeywordTok{read.csv}\NormalTok{(}\StringTok{'C:/Users/mjpearl/Desktop/omsa/MGT-6402-OAN/assignment_3/KAG.csv'}\NormalTok{)}
\KeywordTok{head}\NormalTok{(kag_csv)}
\end{Highlighting}
\end{Shaded}

\begin{verbatim}
##   X  ad_id campaign_id age gender interest Impressions Clicks Spent
## 1 1 708746         916  32      0       15        7350      1  1.43
## 2 2 708749         916  32      0       16       17861      2  1.82
## 3 3 708771         916  32      0       20         693      0  0.00
## 4 4 708815         916  32      0       28        4259      1  1.25
## 5 5 708818         916  32      0       28        4133      1  1.29
## 6 6 708820         916  32      0       29        1915      0  0.00
##   Total_Conversion Approved_Conversion    CTR  CPC CostPerConv_Total
## 1                2                   1 0.0136 1.43             0.715
## 2                2                   0 0.0112 0.91             0.910
## 3                1                   0 0.0000 0.00             0.000
## 4                1                   0 0.0235 1.25             1.250
## 5                1                   1 0.0242 1.29             1.290
## 6                1                   1 0.0000 0.00             0.000
##   CostPerConv_Approved
## 1                 1.43
## 2                 1.82
## 3                 0.00
## 4                 1.25
## 5                 1.29
## 6                 0.00
\end{verbatim}

\hypertarget{including-plots}{%
\subsection{Including Plots}\label{including-plots}}

Q.1 Which ad (provide ad\_id as the answer) among the ads that have the
least CPC led to the most impressions?

In this case I will determine the max impressions based on the CPC = 0.

\begin{Shaded}
\begin{Highlighting}[]
\KeywordTok{library}\NormalTok{(dplyr)}
\end{Highlighting}
\end{Shaded}

\begin{verbatim}
## Warning: package 'dplyr' was built under R version 3.6.3
\end{verbatim}

\begin{verbatim}
## 
## Attaching package: 'dplyr'
\end{verbatim}

\begin{verbatim}
## The following objects are masked from 'package:stats':
## 
##     filter, lag
\end{verbatim}

\begin{verbatim}
## The following objects are masked from 'package:base':
## 
##     intersect, setdiff, setequal, union
\end{verbatim}

\begin{Shaded}
\begin{Highlighting}[]
\KeywordTok{max}\NormalTok{(kag_csv[kag_csv}\OperatorTok{$}\NormalTok{CPC}\OperatorTok{==}\DecValTok{0}\NormalTok{,]}\OperatorTok{$}\NormalTok{Impressions)}
\end{Highlighting}
\end{Shaded}

\begin{verbatim}
## [1] 24362
\end{verbatim}

\begin{Shaded}
\begin{Highlighting}[]
\NormalTok{kag_csv}\OperatorTok{$}\NormalTok{ad_id[kag_csv}\OperatorTok{$}\NormalTok{Impressions}\OperatorTok{==}\DecValTok{24362}\NormalTok{]}
\end{Highlighting}
\end{Shaded}

\begin{verbatim}
## [1] 1121094
\end{verbatim}

Therefore the ad\_id is 1121094.

Q.2 What campaign (provide campaign\_id as the answer) had spent least
efficiently on brand awareness on an average(i.e.~most Cost per mille or
CPM: use total cost for the campaign / total impressions in thousands)?

\begin{Shaded}
\begin{Highlighting}[]
\NormalTok{df_campaign <-}\StringTok{ }\KeywordTok{aggregate}\NormalTok{(.}\OperatorTok{~}\NormalTok{campaign_id, kag_csv, sum)}
\NormalTok{df_campaign}\OperatorTok{$}\NormalTok{output <-}\StringTok{ }\NormalTok{df_campaign}\OperatorTok{$}\NormalTok{Spent}\OperatorTok{/}\NormalTok{df_campaign}\OperatorTok{$}\NormalTok{Impressions}
\NormalTok{df_campaign }\OperatorTok
\StringTok{  }\KeywordTok{select}\NormalTok{(campaign_id, output)}
\end{Highlighting}
\end{Shaded}

\begin{verbatim}
##   campaign_id       output
## 1         916 0.0003100067
## 2         936 0.0003559674
## 3        1178 0.0002717564
\end{verbatim}

As you can see, campaign 936 had the least efficient spending.

Q.3 Assume each conversion (`Total\_Conversion') is worth \$5, each
approved conversion (`Approved\_Conversion') is worth \$50. ROAS (return
on advertising spent) is revenue as a percentage of the advertising
spent . Calculate ROAS and round it to two decimals.

Make a boxplot of the ROAS grouped by gender for interest = 15, 21, 101
(or interest\_id = 15, 21, 101) in one graph. Also try to use the
function `+ scale\_y\_log10()' in ggplot to make the visualization look
better (to do so, you just need to add `+ scale\_y\_log10()' after your
ggplot function). The x-axis label should be `Interest ID' while the
y-axis label should be ROAS. {[}8 points{]}

\begin{Shaded}
\begin{Highlighting}[]
\KeywordTok{library}\NormalTok{(ggplot2)}
\end{Highlighting}
\end{Shaded}

\begin{verbatim}
## Warning: package 'ggplot2' was built under R version 3.6.3
\end{verbatim}

\begin{Shaded}
\begin{Highlighting}[]
\NormalTok{calculate_ROAS <-}\StringTok{ }\ControlFlowTok{function}\NormalTok{(total_conversion, approved_conversion, spent) \{}
    \KeywordTok{round}\NormalTok{(}\DecValTok{100} \OperatorTok{*}\StringTok{ }\NormalTok{(total_conversion }\OperatorTok{*}\StringTok{ }\DecValTok{5} \OperatorTok{+}\StringTok{ }\NormalTok{approved_conversion }\OperatorTok{*}\StringTok{ }\DecValTok{50}\NormalTok{) }\OperatorTok{/}\StringTok{ }\NormalTok{spent, }\DecValTok{2}\NormalTok{)}
\NormalTok{\}}

\NormalTok{kag_csv }\OperatorTok
\StringTok{    }\KeywordTok{select}\NormalTok{(interest, Total_Conversion, Approved_Conversion, Spent, gender) }\OperatorTok\StringTok{ }
\StringTok{    }\KeywordTok{filter}\NormalTok{(interest }\OperatorTok\StringTok{ }\KeywordTok{c}\NormalTok{(}\DecValTok{15}\NormalTok{,}\DecValTok{21}\NormalTok{,}\DecValTok{101}\NormalTok{)) }\OperatorTok\StringTok{ }
\StringTok{    }\KeywordTok{mutate}\NormalTok{(}\DataTypeTok{Gender =} \KeywordTok{factor}\NormalTok{(gender)) }\OperatorTok\StringTok{ }
\StringTok{    }\KeywordTok{mutate}\NormalTok{(}\DataTypeTok{ROAS =} \KeywordTok{calculate_ROAS}\NormalTok{(Total_Conversion, Approved_Conversion, Spent)) }\OperatorTok
\StringTok{    }\KeywordTok{filter}\NormalTok{(}\KeywordTok{is.finite}\NormalTok{(ROAS)) }\OperatorTok\StringTok{ }
\StringTok{    }\KeywordTok{ggplot}\NormalTok{(}\KeywordTok{aes}\NormalTok{(}\DataTypeTok{x =} \KeywordTok{ordered}\NormalTok{(interest), }\DataTypeTok{y =}\NormalTok{ ROAS, }\DataTypeTok{color =}\NormalTok{ Gender)) }\OperatorTok{+}
\StringTok{    }\KeywordTok{geom_boxplot}\NormalTok{() }\OperatorTok{+}
\StringTok{    }\KeywordTok{scale_y_log10}\NormalTok{() }\OperatorTok{+}
\StringTok{    }\KeywordTok{xlab}\NormalTok{(}\StringTok{"Interest ID"}\NormalTok{) }\OperatorTok{+}
\StringTok{    }\KeywordTok{ylab}\NormalTok{(}\StringTok{"ROAS %"}\NormalTok{) }\OperatorTok{+}
\StringTok{    }\KeywordTok{ggtitle}\NormalTok{(}\StringTok{"ROAS % Grouped By Interest, Gender"}\NormalTok{)}
\end{Highlighting}
\end{Shaded}

\includegraphics{hw3-p2_files/figure-latex/q3-1.pdf}

Q.4 Summarize the median and mean of ROAS by genders when campaign\_id
== 1178.

\begin{Shaded}
\begin{Highlighting}[]
\NormalTok{kag_csv }\OperatorTok
\StringTok{    }\KeywordTok{select}\NormalTok{(interest, Total_Conversion, Approved_Conversion, Spent, gender, campaign_id) }\OperatorTok\StringTok{ }
\StringTok{    }\KeywordTok{filter}\NormalTok{(campaign_id }\OperatorTok{==}\StringTok{ }\DecValTok{1178}\NormalTok{) }\OperatorTok\StringTok{ }
\StringTok{    }\KeywordTok{mutate}\NormalTok{(}\DataTypeTok{Gender =} \KeywordTok{factor}\NormalTok{(gender)) }\OperatorTok\StringTok{ }
\StringTok{    }\KeywordTok{mutate}\NormalTok{(}\DataTypeTok{ROAS =} \KeywordTok{calculate_ROAS}\NormalTok{(Total_Conversion, Approved_Conversion, Spent)) }\OperatorTok
\StringTok{    }\KeywordTok{filter}\NormalTok{(}\KeywordTok{is.finite}\NormalTok{(ROAS)) }\OperatorTok
\StringTok{    }\KeywordTok{group_by}\NormalTok{(interest,gender) }\OperatorTok
\StringTok{    }\KeywordTok{summarise}\NormalTok{(}\DataTypeTok{Mean=}\KeywordTok{mean}\NormalTok{(ROAS),}\DataTypeTok{Median=}\KeywordTok{median}\NormalTok{(ROAS))}
\end{Highlighting}
\end{Shaded}

\begin{verbatim}
## # A tibble: 80 x 4
## # Groups:   interest [40]
##    interest gender  Mean Median
##       <int>  <int> <dbl>  <dbl>
##  1        2      0 216.    85.2
##  2        2      1 476.   142. 
##  3        7      0 266.   186. 
##  4        7      1  97.5   45.2
##  5       10      0 120.   103. 
##  6       10      1  91.0   56.5
##  7       15      0 496.   164. 
##  8       15      1 341.   119. 
##  9       16      0 235.   113. 
## 10       16      1  96.7   66.1
## # ... with 70 more rows
\end{verbatim}

\begin{Shaded}
\begin{Highlighting}[]
\NormalTok{adv_csv <-}\StringTok{ }\KeywordTok{read.csv}\NormalTok{(}\StringTok{'C:/Users/mjpearl/Desktop/omsa/MGT-6402-OAN/assignment_3/advertising1.csv'}\NormalTok{)}
\KeywordTok{head}\NormalTok{(adv_csv)}
\end{Highlighting}
\end{Shaded}

\begin{verbatim}
##   Daily.Time.Spent.on.Site Age Area.Income Daily.Internet.Usage
## 1                    68.95  35    61833.90               256.09
## 2                    80.23  31    68441.85               193.77
## 3                    69.47  26    59785.94               236.50
## 4                    74.15  29    54806.18               245.89
## 5                    68.37  35    73889.99               225.58
## 6                    59.99  23    59761.56               226.74
##                           Ad.Topic.Line           City Male    Country
## 1    Cloned 5thgeneration orchestration    Wrightburgh    0    Tunisia
## 2    Monitored national standardization      West Jodi    1      Nauru
## 3      Organic bottom-line service-desk       Davidton    0 San Marino
## 4 Triple-buffered reciprocal time-frame West Terrifurt    1      Italy
## 5         Robust logistical utilization   South Manuel    0    Iceland
## 6       Sharable client-driven software      Jamieberg    1     Norway
##             Timestamp Clicked.on.Ad
## 1 2016-03-27 00:53:11             0
## 2 2016-04-04 01:39:02             0
## 3 2016-03-13 20:35:42             0
## 4 2016-01-10 02:31:19             0
## 5 2016-06-03 03:36:18             0
## 6 2016-05-19 14:30:17             0
\end{verbatim}

Q.5

\begin{enumerate}
\def\labelenumi{\alph{enumi})}
\tightlist
\item
  We aim to explore the dataset so that we can better choose a model to
  implement. Plot histograms for at least 2 of the continuous variables
  in the dataset. Note it is acceptable to plot more than 2. {[}1
  point{]}
\end{enumerate}

\begin{Shaded}
\begin{Highlighting}[]
\NormalTok{adv_csv}\OperatorTok{$}\NormalTok{Clicked.on.Ad <-}\StringTok{ }\KeywordTok{as.factor}\NormalTok{(adv_csv}\OperatorTok{$}\NormalTok{Clicked.on.Ad)}
\KeywordTok{hist}\NormalTok{(adv_csv}\OperatorTok{$}\NormalTok{Area.Income)}
\end{Highlighting}
\end{Shaded}

\includegraphics{hw3-p2_files/figure-latex/q5a-1.pdf}

\begin{Shaded}
\begin{Highlighting}[]
\KeywordTok{hist}\NormalTok{(adv_csv}\OperatorTok{$}\NormalTok{Daily.Internet.Usage)}
\end{Highlighting}
\end{Shaded}

\includegraphics{hw3-p2_files/figure-latex/q5a chart 2-1.pdf}

\begin{enumerate}
\def\labelenumi{\alph{enumi})}
\setcounter{enumi}{1}
\tightlist
\item
  Again on the track of exploring the dataset, plot at least 2 bar
  charts reflecting the counts of different values for different
  variables. Note it is acceptable to plot more than 2. {[}1 point{]}
\end{enumerate}

\begin{Shaded}
\begin{Highlighting}[]
\KeywordTok{barplot}\NormalTok{(}\KeywordTok{table}\NormalTok{(adv_csv}\OperatorTok{$}\NormalTok{Age), }\DataTypeTok{main=}\StringTok{"Age Value Count"}\NormalTok{,}
   \DataTypeTok{xlab=}\StringTok{"Age"}\NormalTok{)}
\end{Highlighting}
\end{Shaded}

\includegraphics{hw3-p2_files/figure-latex/q5b-1.pdf}

\begin{Shaded}
\begin{Highlighting}[]
\KeywordTok{barplot}\NormalTok{(}\KeywordTok{table}\NormalTok{(adv_csv}\OperatorTok{$}\NormalTok{Country), }\DataTypeTok{main=}\StringTok{"Country Value Count"}\NormalTok{,}
   \DataTypeTok{xlab=}\StringTok{"Country"}\NormalTok{)}
\end{Highlighting}
\end{Shaded}

\includegraphics{hw3-p2_files/figure-latex/q5c-1.pdf} c) Plot boxplots
for Age, Area.Income, Daily.Internet.Usage and Daily.Time.Spent.on.Site
separated by the variable Clicked.on.Ad. To clarify, we want to create 4
plots, each of which has 2 boxplots: 1 for people who clicked on the ad,
one for those who didn't. {[}2 points{]}

\begin{Shaded}
\begin{Highlighting}[]
\NormalTok{adv_csv }\OperatorTok
\StringTok{    }\KeywordTok{select}\NormalTok{(Age, Area.Income, Daily.Internet.Usage, Daily.Time.Spent.on.Site, Clicked.on.Ad) }\OperatorTok
\StringTok{    }\KeywordTok{ggplot}\NormalTok{(}\KeywordTok{aes}\NormalTok{(}\DataTypeTok{x =} \KeywordTok{ordered}\NormalTok{(Clicked.on.Ad), }\DataTypeTok{y =}\NormalTok{ Age, }\DataTypeTok{color =}\NormalTok{ Clicked.on.Ad)) }\OperatorTok{+}
\StringTok{    }\KeywordTok{geom_boxplot}\NormalTok{() }\OperatorTok{+}
\StringTok{    }\KeywordTok{xlab}\NormalTok{(}\StringTok{"Clicked On Ad"}\NormalTok{) }\OperatorTok{+}
\StringTok{    }\KeywordTok{ylab}\NormalTok{(}\StringTok{"Age"}\NormalTok{) }\OperatorTok{+}
\StringTok{    }\KeywordTok{ggtitle}\NormalTok{(}\StringTok{"Age Grouped by Click On Ad"}\NormalTok{)}
\end{Highlighting}
\end{Shaded}

\includegraphics{hw3-p2_files/figure-latex/q5c 1-1.pdf}

\begin{Shaded}
\begin{Highlighting}[]
\NormalTok{adv_csv }\OperatorTok
\StringTok{    }\KeywordTok{select}\NormalTok{(Age, Area.Income, Daily.Internet.Usage, Daily.Time.Spent.on.Site, Clicked.on.Ad) }\OperatorTok
\StringTok{    }\KeywordTok{ggplot}\NormalTok{(}\KeywordTok{aes}\NormalTok{(}\DataTypeTok{x =} \KeywordTok{ordered}\NormalTok{(Clicked.on.Ad), }\DataTypeTok{y =}\NormalTok{ Area.Income, }\DataTypeTok{color =}\NormalTok{ Clicked.on.Ad)) }\OperatorTok{+}
\StringTok{    }\KeywordTok{geom_boxplot}\NormalTok{() }\OperatorTok{+}
\StringTok{    }\KeywordTok{xlab}\NormalTok{(}\StringTok{"Clicked On Ad"}\NormalTok{) }\OperatorTok{+}
\StringTok{    }\KeywordTok{ylab}\NormalTok{(}\StringTok{"Income"}\NormalTok{) }\OperatorTok{+}
\StringTok{    }\KeywordTok{ggtitle}\NormalTok{(}\StringTok{"Income Grouped by Click On Ad"}\NormalTok{)}
\end{Highlighting}
\end{Shaded}

\includegraphics{hw3-p2_files/figure-latex/q5c 2-1.pdf}

\begin{Shaded}
\begin{Highlighting}[]
\NormalTok{adv_csv }\OperatorTok
\StringTok{    }\KeywordTok{select}\NormalTok{(Age, Area.Income, Daily.Internet.Usage, Daily.Time.Spent.on.Site, Clicked.on.Ad) }\OperatorTok
\StringTok{    }\KeywordTok{ggplot}\NormalTok{(}\KeywordTok{aes}\NormalTok{(}\DataTypeTok{x =} \KeywordTok{ordered}\NormalTok{(Clicked.on.Ad), }\DataTypeTok{y =}\NormalTok{ Daily.Time.Spent.on.Site, }\DataTypeTok{color =}\NormalTok{ Clicked.on.Ad)) }\OperatorTok{+}
\StringTok{    }\KeywordTok{geom_boxplot}\NormalTok{() }\OperatorTok{+}
\StringTok{    }\KeywordTok{xlab}\NormalTok{(}\StringTok{"Clicked On Ad"}\NormalTok{) }\OperatorTok{+}
\StringTok{    }\KeywordTok{ylab}\NormalTok{(}\StringTok{"Daily.Time.Spent.on.Site"}\NormalTok{) }\OperatorTok{+}
\StringTok{    }\KeywordTok{ggtitle}\NormalTok{(}\StringTok{"Daily Time Spent on Site Grouped by Click On Ad"}\NormalTok{)}
\end{Highlighting}
\end{Shaded}

\includegraphics{hw3-p2_files/figure-latex/q5c 3-1.pdf}

\begin{Shaded}
\begin{Highlighting}[]
\NormalTok{adv_csv }\OperatorTok
\StringTok{    }\KeywordTok{select}\NormalTok{(Age, Area.Income, Daily.Internet.Usage, Daily.Time.Spent.on.Site, Clicked.on.Ad) }\OperatorTok
\StringTok{    }\KeywordTok{ggplot}\NormalTok{(}\KeywordTok{aes}\NormalTok{(}\DataTypeTok{x =} \KeywordTok{ordered}\NormalTok{(Clicked.on.Ad), }\DataTypeTok{y =}\NormalTok{ Daily.Internet.Usage, }\DataTypeTok{color =}\NormalTok{ Clicked.on.Ad)) }\OperatorTok{+}
\StringTok{    }\KeywordTok{geom_boxplot}\NormalTok{() }\OperatorTok{+}
\StringTok{    }\KeywordTok{xlab}\NormalTok{(}\StringTok{"Clicked On Ad"}\NormalTok{) }\OperatorTok{+}
\StringTok{    }\KeywordTok{ylab}\NormalTok{(}\StringTok{"Daily.Internet.Usage"}\NormalTok{) }\OperatorTok{+}
\StringTok{    }\KeywordTok{ggtitle}\NormalTok{(}\StringTok{"Daily Internet Usage Grouped by Click On Ad"}\NormalTok{)}
\end{Highlighting}
\end{Shaded}

\includegraphics{hw3-p2_files/figure-latex/q5c 4-1.pdf} d) Based on our
preliminary boxplots, would you expect an older person to be more likely
to click on the ad than someone younger? {[}2 points{]}

Yes based on the results you can see that with increasing Age, people
click more on the advertisement compared to the younger generation.

Q.6

Part (a) {[}3 points{]}

\begin{enumerate}
\def\labelenumi{\arabic{enumi}.}
\tightlist
\item
  Make a scatter plot for Area.Income against Age. Separate the
  datapoints by different shapes based on if the datapoint has clicked
  on the ad or not.
\end{enumerate}

\includegraphics{hw3-p2_files/figure-latex/q6_a-1.pdf}

\begin{enumerate}
\def\labelenumi{\arabic{enumi}.}
\setcounter{enumi}{1}
\tightlist
\item
  Based on this plot, would you expect a 31-year-old person with an Area
  income of \$62,000 to click on the ad or not?
\end{enumerate}

You can see based on the plot you can see that it's highly likely the
person matching these conditions did not click on the Ad.

Part (b) {[}3 points{]}

\begin{enumerate}
\def\labelenumi{\arabic{enumi}.}
\tightlist
\item
  Similar to part a), create a scatter plot for Daily.Time.Spent.on.Site
  against Age. Separate the datapoints by different shapes based on if
  the datapoint has clicked on the ad or not.
\end{enumerate}

\includegraphics{hw3-p2_files/figure-latex/q6_b-1.pdf}

\begin{enumerate}
\def\labelenumi{\arabic{enumi}.}
\setcounter{enumi}{1}
\tightlist
\item
  Based on this plot, would you expect a 50-year-old person who spends
  60 minutes daily on the site to click on the ad or not?
\end{enumerate}

Yes you could still likely say that they would click on the ad. However,
this seems to be very close to the cut-off point where most observations
start to become 0 or no for Clicked on Ad.

Q.7

Part (a) {[}2 points{]}

\begin{enumerate}
\def\labelenumi{\arabic{enumi}.}
\item
  Now that we have done some exploratory data analysis to get a better
  understanding of our raw data, we can begin to move towards designing
  a model to predict advert clicks.
\item
  Generate a correlation funnel (using the correlation funnel package)
  to see which of the variable in the dataset have the most correlation
  with having clicked the advert.
\end{enumerate}

\begin{Shaded}
\begin{Highlighting}[]
\KeywordTok{library}\NormalTok{(correlationfunnel)}
\end{Highlighting}
\end{Shaded}

\begin{verbatim}
## Warning: package 'correlationfunnel' was built under R version 3.6.3
\end{verbatim}

\begin{verbatim}
## == correlationfunnel Tip #3 ================================================================================================
## Using `binarize()` with data containing many columns or many rows can increase dimensionality substantially.
## Try subsetting your data column-wise or row-wise to avoid creating too many columns.
## You can always make a big problem smaller by sampling. :)
\end{verbatim}

\begin{Shaded}
\begin{Highlighting}[]
\KeywordTok{library}\NormalTok{(dplyr)}

\NormalTok{adv_csv_binarized_tbl <-}\StringTok{ }\NormalTok{adv_csv }\OperatorTok
\StringTok{  }\KeywordTok{mutate}\NormalTok{(}\DataTypeTok{Age=} \KeywordTok{as.numeric}\NormalTok{(Age),}
         \DataTypeTok{Male =} \KeywordTok{factor}\NormalTok{(Male)) }\OperatorTok\StringTok{ }
\StringTok{  }\KeywordTok{binarize}\NormalTok{(}\DataTypeTok{n_bins=}\DecValTok{5}\NormalTok{, }\DataTypeTok{thresh_infreq =} \FloatTok{0.01}\NormalTok{, }\DataTypeTok{name_infreq =} \StringTok{"OTHER"}\NormalTok{, }\DataTypeTok{one_hot =} \OtherTok{TRUE}\NormalTok{)}

\NormalTok{adv_csv_corr_tbl <-}\StringTok{ }\NormalTok{adv_csv_binarized_tbl }\OperatorTok
\StringTok{  }\KeywordTok{correlate}\NormalTok{(Clicked.on.Ad__}\DecValTok{1}\NormalTok{)}

\NormalTok{adv_csv_corr_tbl }\OperatorTok
\StringTok{  }\KeywordTok{arrange}\NormalTok{(}\KeywordTok{desc}\NormalTok{(correlation)) }\OperatorTok
\StringTok{  }\KeywordTok{plot_correlation_funnel}\NormalTok{()}
\end{Highlighting}
\end{Shaded}

\includegraphics{hw3-p2_files/figure-latex/q7a-1.pdf} From the
correlation plot, we can see that the first 4 varaibles containing the
highest correlation to Click.on.Ad is

\begin{Shaded}
\begin{Highlighting}[]
\CommentTok{#Let's retrieve the 4 highest correlated variables to use for the logistic regression}
\NormalTok{adv_csv_corr_tbl}
\end{Highlighting}
\end{Shaded}

\begin{verbatim}
## # A tibble: 32 x 3
##    feature                  bin             correlation
##    <fct>                    <chr>                 <dbl>
##  1 Clicked.on.Ad            0                    -1    
##  2 Clicked.on.Ad            1                     1    
##  3 Daily.Time.Spent.on.Site -Inf_47.23            0.502
##  4 Daily.Internet.Usage     -Inf_132.366          0.5  
##  5 Daily.Internet.Usage     132.366_163.44        0.445
##  6 Daily.Internet.Usage     224.836_Inf          -0.44 
##  7 Daily.Time.Spent.on.Site 79.982_Inf           -0.42 
##  8 Area.Income              -Inf_43644.412        0.41 
##  9 Daily.Internet.Usage     198.948_224.836      -0.41 
## 10 Daily.Time.Spent.on.Site 47.23_62.26           0.395
## # ... with 22 more rows
\end{verbatim}

NOTE: Here we are creating the correlation funnel in regards to HAVING
clicked the advert, rather than not. This will lead to a minor
distinction in your code between the 2 cases. However, it will not
affect your results and subsequent variable selection. Part (b) {[}2
points{]}

\begin{enumerate}
\def\labelenumi{\arabic{enumi}.}
\tightlist
\item
  Based on the generated correlation funnel, choose the 4 most covarying
  variables (with having clicked the advert) and run a logistic
  regression model for Clicked.on.Ad using these 4 variables.
\end{enumerate}

\begin{Shaded}
\begin{Highlighting}[]
\CommentTok{#Let's retrieve the 4 highest correlated variables to use for the logistic regression}

\NormalTok{logit <-}\StringTok{ }\KeywordTok{glm}\NormalTok{(Clicked.on.Ad }\OperatorTok{~}\StringTok{ `}\DataTypeTok{Daily.Time.Spent.on.Site}\StringTok{`} \OperatorTok{+}\StringTok{ `}\DataTypeTok{Daily.Internet.Usage}\StringTok{`} \OperatorTok{+}\StringTok{ `}\DataTypeTok{Area.Income}\StringTok{`} \OperatorTok{+}\StringTok{ `}\DataTypeTok{Age}\StringTok{`}\NormalTok{, }\DataTypeTok{data=}\NormalTok{adv_csv,}
             \DataTypeTok{family =} \StringTok{"binomial"}\NormalTok{)}
\end{Highlighting}
\end{Shaded}

\begin{enumerate}
\def\labelenumi{\arabic{enumi}.}
\setcounter{enumi}{1}
\tightlist
\item
  Output the summary of this model.
\end{enumerate}

\begin{Shaded}
\begin{Highlighting}[]
\KeywordTok{summary}\NormalTok{(logit)}
\end{Highlighting}
\end{Shaded}

\begin{verbatim}
## 
## Call:
## glm(formula = Clicked.on.Ad ~ Daily.Time.Spent.on.Site + Daily.Internet.Usage + 
##     Area.Income + Age, family = "binomial", data = adv_csv)
## 
## Deviance Residuals: 
##     Min       1Q   Median       3Q      Max  
## -2.4578  -0.1341  -0.0333   0.0167   3.1961  
## 
## Coefficients:
##                            Estimate Std. Error z value Pr(>|z|)    
## (Intercept)               2.713e+01  2.714e+00   9.995  < 2e-16 ***
## Daily.Time.Spent.on.Site -1.919e-01  2.066e-02  -9.291  < 2e-16 ***
## Daily.Internet.Usage     -6.391e-02  6.745e-03  -9.475  < 2e-16 ***
## Area.Income              -1.354e-04  1.868e-05  -7.247 4.25e-13 ***
## Age                       1.709e-01  2.568e-02   6.655 2.83e-11 ***
## ---
## Signif. codes:  0 '***' 0.001 '**' 0.01 '*' 0.05 '.' 0.1 ' ' 1
## 
## (Dispersion parameter for binomial family taken to be 1)
## 
##     Null deviance: 1386.3  on 999  degrees of freedom
## Residual deviance:  182.9  on 995  degrees of freedom
## AIC: 192.9
## 
## Number of Fisher Scoring iterations: 8
\end{verbatim}

Q.8 {[}4 points{]}

Now that we have created our logistic regression model using variables
of significance, we must test the model. When testing such models, it is
always recommended to split the data into a training (from which we
build the model) and test (on which we test the model) set. This is done
to avoid bias, as testing the model on the data from which it is
originally built from is unrepresentative of how the model will perform
on new data. That said, for the case of simplicity, test the model on
the full original dataset. Use type =``response'' to ensure we get the
predicted probabilities of clicking the advert Append the predicted
probabilities to a new column in the original dataset or simply to a new
data frame. The choice is up to you, but ensure you know how to
reference this column of probabilities. Using a threshold of 80\% (0.8),
create a new column in the original dataset that represents if the model
predicts a click or not for that person. Note this means probabilities
above 80\% should be treated as a click prediction. Now using the caret
package, create a confusion matrix for the model predictions and actual
clicks. Note you do not need to graph or plot this confusion matrix. How
many false-negative occurrences do you observe? Recall false negative
means the instances where the model predicts the case to be false when
in reality it is true. For this example, this refers to cases where the
ad is clicked but the model predicts that it isn't

\begin{Shaded}
\begin{Highlighting}[]
\KeywordTok{library}\NormalTok{(ROCR)}
\end{Highlighting}
\end{Shaded}

\begin{verbatim}
## Loading required package: gplots
\end{verbatim}

\begin{verbatim}
## 
## Attaching package: 'gplots'
\end{verbatim}

\begin{verbatim}
## The following object is masked from 'package:stats':
## 
##     lowess
\end{verbatim}

\begin{Shaded}
\begin{Highlighting}[]
\NormalTok{predictLogit <-}\StringTok{ }\KeywordTok{predict}\NormalTok{(logit, }\DataTypeTok{type=}\StringTok{'response'}\NormalTok{)}
\NormalTok{adv_csv}\OperatorTok{$}\NormalTok{output <-}\StringTok{ }\NormalTok{predictLogit}

\KeywordTok{head}\NormalTok{(adv_csv)}
\end{Highlighting}
\end{Shaded}

\begin{verbatim}
##   Daily.Time.Spent.on.Site Age Area.Income Daily.Internet.Usage
## 1                    68.95  35    61833.90               256.09
## 2                    80.23  31    68441.85               193.77
## 3                    69.47  26    59785.94               236.50
## 4                    74.15  29    54806.18               245.89
## 5                    68.37  35    73889.99               225.58
## 6                    59.99  23    59761.56               226.74
##                           Ad.Topic.Line           City Male    Country
## 1    Cloned 5thgeneration orchestration    Wrightburgh    0    Tunisia
## 2    Monitored national standardization      West Jodi    1      Nauru
## 3      Organic bottom-line service-desk       Davidton    0 San Marino
## 4 Triple-buffered reciprocal time-frame West Terrifurt    1      Italy
## 5         Robust logistical utilization   South Manuel    0    Iceland
## 6       Sharable client-driven software      Jamieberg    1     Norway
##             Timestamp Clicked.on.Ad      output
## 1 2016-03-27 00:53:11             0 0.007678840
## 2 2016-04-04 01:39:02             0 0.009738814
## 3 2016-03-13 20:35:42             0 0.006892949
## 4 2016-01-10 02:31:19             0 0.005057957
## 5 2016-06-03 03:36:18             0 0.011744099
## 6 2016-05-19 14:30:17             0 0.045802993
\end{verbatim}

We can see the results from the output of the original dataset that the
prediction colum ``output'' has been added as a column.

\begin{Shaded}
\begin{Highlighting}[]
\KeywordTok{table}\NormalTok{(adv_csv}\OperatorTok{$}\NormalTok{Clicked.on.Ad,adv_csv}\OperatorTok{$}\NormalTok{output }\OperatorTok{>}\StringTok{ }\FloatTok{0.8}\NormalTok{)}
\end{Highlighting}
\end{Shaded}

\begin{verbatim}
##    
##     FALSE TRUE
##   0   497    3
##   1    36  464
\end{verbatim}

From our table result we can see that the False Negative in our case is
3. Or in other words, this refers to 3 cases where the ad is clicked but
the model predicts that it isn't.

\end{document}
